\section{NOE: NOE and distance restraints term (\texttt{noe})}
\label{sec:noe}

This energy function handles distance restraints, such as from NOE experiments.
The code has been developed with sparse NOE restraints in large proteins $> 100-200$ residues in mind, but also handles other types of distance restraints well, such as for instance distance restraints from co-evolutionary sequence analysis \cite{10.7287/peerj.preprints.374v1}.
The energy function is a flat-bottom potential with a linear increase at large distances, as used in the RASREC-ROSETTA protocol \cite{LangePNAS2012}. The functional form is:
\begin{equation}
   f(x) =
\begin{cases}
\left( \frac{x-lb}{sd} \right)^2        & \text{for } x < lb \\
0                                       & \text{for } lb < x \leq ub \\
\left( \frac{x-ub}{sd} \right)^2        & \text{for } ub < x \leq ub + rswitch \cdot sd \\
\frac{1}{sd}\left(x - \left( ub + rswitch \cdot sd \right) \right) 
+ \left( \frac{rswitch \cdot sd }{sd} \right)^2
                                        & \text{for } x > ub + rswitch \cdot sd
\end{cases}
\end{equation}
where $ub$ is the measured distance restraint, $x$ is the actual value in the structure, 
$lb = 1.5$ \AA, $sd = 0.3$ \AA, and $rswitch = 0.5$ \AA.
\\\\To avoid getting trapped in local minima, only a fraction of all the restraints are used at a time.
This is done to allow the structure to easily get out of conformations where restraints constrain the structure to a misfolded, non-native conformation which may satisfies several constraints.

Restraints can be switched between on and off states randomly (default behavior), or according to the Boltzmann distribution (or multi-canonical distribution if MUNINN is used).

The number of concurrently active restraints is constant and set by the user.
Suggested numbers could be 20-30\% of all restraints for a folding simulation (from unfolded), and 75-100\% of all restraints for a refinement simulation.
The optimal choice may vary a lot from protein to protein and from one set of settings to another.
\\\\The input format is CYANAs .UPL format, e.g.:
\begin{verbatim}
153     SER     H   157     LYS     H       4.750
 15     MET     H    48     ALA     QB      4.770
 16     LEU     H    48     ALA     QB      4.610
 16     LEU     CD   48     ALA     CB      7.200
...
\end{verbatim}
The \texttt{Qx} notation denotes ambiguous restraints, e.g. \texttt{QB} on an Alanine residue denotes that the restraint is from any \texttt{HB} atom.
The code checks and verifies for correct residue types.
This check can be disabled by setting the residue type to \texttt{"UNK"} in the .UPL file. 
\\\\If used as an observable, the energy reported corresponds to the energy contribution from ALL restraints (i.e. not only the active subset).
Furthermore, the number of restraints that are satisfied is reported as well as the number of restraints within one and five \AA ngstr\o m, respectively, from satisfying the restraint.
Lastly the number of restraints not satisfying any of these criteria is reported.

\subsubsection*{NOE energy and MUNINN sampling}
If random switching of restraints is selected and sampling is performed from the multi-canonical distribution through MUNINN, it is advised to use the NOE energy term as a secondary energy term outside MUNINN. 
This is achieved by using the following keyword \texttt{energy2} to enable the NOE term, and by enable the use of a secondary energy in MUNINN:
\begin{verbatim}
monte-carlo-muninn-use-energy2 = 1
energy2-noe = 1
\end{verbatim}
\optiontitle{Settings}
\begin{optiontable}
  \option{active-restraints}{int}{0}{The number of simultaneously active restraints during simulation.}
  \option{seamless}{bool}{True}{Force the energy difference of a change in restraints to be zero by adding a bias (i.e. randomly switch stat of a given restraint).}
  \option{upl-filename}{string}{default}{The name of the CYANA .UPL-formatted file containing the restraints.}
\end{optiontable}
